\subsection*{Definiciones recursivas}
Supongamos que queremos definir una función 
\[ H: \p \rightarrow A \]
donde $A$ puede tomar diferentes tipos de conjunto: $\mathbb{N}$, $\mathcal{P}(\p)$,... para hacerlo de forma recursiva, vamos a necesitar las siguientes funciones auxiliares
\begin{itemize}
	\item Caso atómico
	\[ H_{\mbox{AT}}: \mbox{AT} \rightarrow A \]
dada por $H(\varphi)=H_{\mbox{AT}}(\varphi), \forall \varphi \in \mbox{AT}$
	\item Paso recursivo: donde diferenciamos entre la negación y las conectivas binarias. 
	\begin{itemize}
		\item[(i)] Negación 
		\[ H_{\lnot}: A \rightarrow A \]
		dada por 
	\[ H \vert (\lnot \varphi) \vert = H_{\lnot}\vert (H(\varphi) \vert \]
		\item[(ii)] Conectivas binarias 
		\[ H_{\Box}: A \times A \rightarrow A \]
		dada por 
	\[ H \vert (\varphi_1 \boox \varphi_2) \vert = H_{\Box}\vert (H(\varphi_1, \, \varphi_2) \vert \]
	\end{itemize}
\end{itemize} 
En los casos (i) e (ii) ni la entrada ni la salida de la función está formada por formulas. 
\paragraph{}
\addtocounter{ej}{1} % sumo 1
\textbf{Ejemplo \Roman{ej}}: Si queremos definir de forma recursiva el número de paréntesis abiertos de una proposición
\[ H_{(}: \p \rightarrow \mathbb{N} \] 
se tendría
\[ H_{\mbox{AT}\,(}: \mbox{AT} \rightarrow \mathbb{N} \]
dada por
\[ H_{\mbox{AT}\,(}(\varphi)=0, \forall \varphi \in \mbox{AT} \]
la negación
\[ \begin{matrix}
 H_{\lnot \,(}: & \mathbb{N} & \rightarrow & \mathbb{N}\\
 &n& \longmapsto &  n+1
\end{matrix} \] 
finalmente, el resto de conectivas 
\[ \begin{matrix}
 H_{\Box \,(} :  & \mathbb{N} \times \mathbb{N} & \rightarrow & \mathbb{N}\\
 &(n,\, m)& \longmapsto &  n+m+1
\end{matrix} \] 
Escribirlo así es un poco farragoso por lo que, generalmente, escribiremos las funciones haciendo uso del propio concepto de recursión
\[ H_{(}(\varphi)=0, \forall \varphi \in \mbox{AT} \]
\[ H_{\lnot \,(}((\neg \varphi))=H_{(}(\varphi) +1 \]
\[ H_{\Box \,(}(\varphi_1 \boox \varphi_2)= H_{(}(\varphi_1)+H_{(}(\varphi_2)+1 \]
además, en la escritura, se suprime la función auxiliar. 
\addtocounter{ej}{1} % sumo 1
\textbf{Ejemplo \Roman{ej}}: definiremos de forma recursiva el número de apariciones de $\wedge$ en $\varphi$ y lo denotamos como $\vert \varphi \vert^{\wedge}$.
\begin{itemize}
	\item[(AT)] $\vert \varphi \vert^{\wedge}=0, \forall \varphi \in \mbox{AT}$
	\item[($\neg$)] $\vert (\neg \varphi) \vert^{\wedge}=\vert \varphi \vert^{\wedge}$
	\item[($\Box$)] Hay que diferenciar dos casos
	\[ \vert (\varphi_1 \boox \varphi_2) \vert^{\wedge}= \left\lbrace \begin{matrix}
	\vert \varphi_1 \vert^{\wedge}+\vert \varphi_2 \vert^{\wedge}& \mbox{si}& \Box \in \{ \lor, \, \rightarrow, \, \leftrightarrow\}\\
	\\
	\varphi_1 \vert^{\wedge}+\vert \varphi_2 \vert^{\wedge}+1 & \mbox{si}& \Box=\wedge
\end{matrix} \right.	 \] 
\end{itemize}
\begin{definition} Dadas dos fórmulas $\varphi$ y $\psi$ decimos que $\psi$ es una \textbf{subformula} de $\varphi$ si una parte de $\varphi$ formada por símbolos consecutivos es idéntica a $\psi$.
\end{definition}
\addtocounter{ej}{1} % sumo 1
\textbf{Ejemplo \Roman{ej}}: dada la formula, 
\[ \varphi \equiv (p \lor (q \rightarrow (\neg r))) \]
observamos que esta formada por las subformulas siguientes
\[ p, \, q, \, r \, (\neg r), \, (q \rightarrow (\neg r)), \varphi \]
\addtocounter{ej}{1} % sumo 1
\textbf{Ejemplo \Roman{ej}}: Creamos una función recursiva que devuelva las diferentes subformulas de una formula dada.
\[ \mbox{SUB}: \p \rightarrow \mathcal{P}(\p) \]
dada por 
\begin{itemize}
	\item[(AT)] $\mbox{SUB}(\varphi)=\{\varphi \}, \forall \varphi \in \mbox{AT}$
	\item[($\neg$)] $\mbox{SUB}((\neg \varphi))=\{ (\neg\varphi)\} \cup \mbox{SUB}(\varphi)$
	\item[($\Box$)] $ \mbox{SUB} ((\varphi_1 \boox \varphi_2) )= \mbox{SUB}(\varphi_1)+\mbox{SUB}(\varphi_2) \cup \{ (\varphi_1 \boox \varphi_2) \}$
\end{itemize}
\begin{prop} El esquema de definición recursiva da como resultado una única función. Esto es, dadas 
\[ H_{\mbox{AT}}: \mbox{AT} \rightarrow A \]
\[ H_{\lnot}: A \rightarrow A \]
\[ H_{\Box}: A \times A \rightarrow A \]
existe una única función $H: \p \rightarrow A$ que verifica 
\[ H(\varphi)=H_{\mbox{AT}}(\varphi), \forall \varphi \in \mbox{AT} \]
\[H((\neg \varphi))=H_{\neg}(H (\varphi))\]
\[H ( (\varphi_1 \boox \varphi_2) ) = H_{\Box}(H(\varphi_1, \, \varphi_2))\]
\end{prop}
\subsection*{Eliminación de paréntesis}
Como en la aritmética básica, cuando escribimos una operación podemos utilizar las reglas de prioridad, asociatividad, etc. para escribir el menor número de paréntesis posible. Así por ejemplo en 
\[ (((2 \cdot 3)+5)-(3 \cdot 2))=2 \cdot 3 +5-3\cdot 3 \]
la idea es, por tanto, dar una serie de reglas para poder hacer lo mismo con nuestras formulas de proposición. 
\subsubsection*{Reglas de eliminación de paréntesis}
\begin{enumerate}
	\item \textbf{Elminación de paréntesis externos}. No aportan información.
	\item \textbf{Prioridad entre conectivas}. En la siguiente lista, las conectivas, aparecen de más prioridad a menos
	\[ (+)\quad \neg, \, \wedge, \, \lor, \, \rightarrow, \, \leftrightarrow \quad (-) \]
	\item \textbf{Asociatividad}. Adoptamos el convenio de asociar por la izquierda. De este modo si tenemos $p \rightarrow q \rightarrow r$ daremos como asociación válida 
	\[ (p \rightarrow q) \rightarrow r \]
	siendo, por tanto, errónea 
	\[ p \rightarrow (q \rightarrow r)  \]
	Esto mismo se aplica para $\leftrightarrow$. En los casos de $\vee$ y $\wedge$ no se presenta problema pues son asociativas.
\end{enumerate}