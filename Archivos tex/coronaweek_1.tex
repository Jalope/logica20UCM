\begin{definition}
Conjunto de constantes, símbolos de función y símbolos de predicado que aparecen en términos y fórmulas. 
\end{definition}

\begin{definition}
Sea $S= \si{S}$ una signatura. El vocabulario \textbf{para términos} se define recursivamente como
\begin{itemize}
	\item Base 
	\begin{itemize}
		\item $\mbox{voc}(c)=\{ c \}, \quad c \in \cts{S}$ 
		\item $\mbox{voc}(x)=\{ \emptyset \}, \quad x \in \var$ 
	\end{itemize}
	\item Recursivo 
	\begin{itemize}
		\item $\mbox{voc}(f(t_1, \ldots, t_n))=\{ f \} \cup \bigcup_{i=1}^k \mbox{voc}(t_i), \quad f\vert_{k} \in \fn{S} \quad t_i \in \term{S}$ 
	\end{itemize}
\end{itemize} 
\textbf{para fórmulas} se define como
\begin{itemize}
	\item Base
	\begin{itemize}
		\item $\mbox{voc}(p(t_1,\ldots, t_k))= \{p\} \cup \bigcup_{i=1}^k \mbox{voc}(t_i), \quad p \in \pd{S} \quad t_i \in \term{S}$
		\item $\mbox{voc}(t_1 \doteq t_2)= \mbox{voc}(t_1) \cup \mbox{voc}(t_2) \quad t_1,\, t_2 \in \term{S}$
		\item $\mbox{voc}(\top)=\mbox{voc}(\bot)=\emptyset$
	\end{itemize}
	\item Recursivo
	\begin{itemize}
		\item $\mbox{voc}(\neg \varphi)=\mbox{voc}(\varphi), \quad \varphi \in \form{S}$
		\item $\mbox{voc}(\varphi \boox \psi) = \mbox{voc}(\varphi) \cup \mbox{voc}(\psi), \quad \varphi, \, \psi \in \form{S}$
		\item $\mbox{voc}(Qx\varphi)=\mbox{voc}(\varphi), \quad Q \in \{\exists, \, \forall\}, \quad x \in \var \quad \varphi \in \form{S}$
	\end{itemize}
\end{itemize}
\end{definition}

\textbf{Notación}. Sean $S_1$ y $S_2$ son signaturas tal que 
\[ S_1= \si{S_1} \]
\[ S_2= \si{S_2} \]
entonces 
\[ S_1 \cup S_2 = \{ \cts{S_1} \cup \cts{S_2}, \, \fn{S_1} \cup \fn{S_2}, \, \pd{S_1} \cup \pd{S_2} \} \]
\[ S_1 \cap S_2 = \{ \cts{S_1} \cap \cts{S_2}, \, \fn{S_1} \cap \fn{S_2}, \, \pd{S_1} \cap \pd{S_2} \} \]

\begin{definition}
Sean $S_1$ y $S_2$ dos signaturas $\mathfrak{Y}_1=\langle \mathfrak{a}_1, \, \sigma_1 \rangle$ una $s_1$-interpretación e $\mathfrak{Y}_2=\langle \mathfrak{a}_2, \, \sigma_2 \rangle$ una $s_2$-interpretación. $\mathfrak{Y}_1$ e $\mathfrak{Y}_2$ tienen el mismo soporte. Sea $S=S_1 \cap S_2$, $t \in \term{S}$ y $\varphi \in \form{S}$
\begin{itemize}
	\item $\mathfrak{Y}_1$ e $\mathfrak{Y}_2$ coinciden en $t$ $(\mathfrak{Y}_1 \sim_{t} \mathfrak{Y}_2)$ si 
	\begin{itemize}
		\item[a)] Para todo $c \in \mbox{voc}(t), \quad c^{\mathfrak{Y}_1}=c^{\mathfrak{Y}_2}$
		\item[b)] Para todo $x \in \var(t), \quad x^{\mathfrak{Y}_1}=x^{\mathfrak{Y}_2}$
		\item[c)] Para todo $f \in \mbox{voc}(t), \quad f \in \fn{S}, \quad f^{\mathfrak{a}_1}=f^{\mathfrak{a}_2}$
\end{itemize}	 
\item $\mathfrak{Y}_1$ e $\mathfrak{Y}_2$ coinciden en $\varphi$ $(\mathfrak{Y}_1 \sim_{\varphi} \mathfrak{Y}_2)$ si
\begin{itemize}
	\item[a)] Para todo $c \in \mbox{voc}(\varphi), \quad c \in \cts{S}: \; c^{\mathfrak{a}_1}=c^{\mathfrak{a}_2}$
	\item[b)] Para todo $x \in \mbox{lib}(\varphi): \; \sigma_1(x)=\sigma_2(x)$
	\item[c)] Para todo $f \in \mbox{voc}(\varphi), \quad f \in \fn{S}: \; f^{\mathfrak{a}_1}=f^{\mathfrak{a}_2}$
	\item[d)] Para todo $p \in \mbox{voc}(\varphi), \quad p \in \pd{S}: \; p^{\mathfrak{a}_1}=p^{\mathfrak{a}_2}$
\end{itemize}
\end{itemize} 
\end{definition}

\begin{theorem}[\textit{Lema de coincidencia}]
Sean $S_1$ y $S_2$ dos signaturas, $\mathfrak{Y}_1$ una $S_1$-interpretación e $\mathfrak{Y}_2$ una $S_2$-interpretación con el mismo soporte y $S=S_1 \cap S_2$. 
\begin{enumerate}
	\item $t \in \term{S}$ tal que $\mathfrak{Y}_1 \sim_{t} \mathfrak{Y}_2$ entonces 
	\[ t^{\mathfrak{Y}_1}=t^{\mathfrak{Y}_2} \]
	\item $\varphi \in \form{S}$ tal que $\mathfrak{Y}_1 \sim_{\varphi} \mathfrak{Y}_2$ entonces 
	\[ \varphi^{\mathfrak{Y}_1}=\varphi^{\mathfrak{Y}_2} \]
\end{enumerate}
\end{theorem}
\begin{proof}
{\color{blue}\textit{Esta en el foro, cuando Luis de el OK, la añado a los apuntes}.}
\end{proof}

\textbf{Notación}. El lema de coincidencia nos permite ignorar el significado de los símbolos que no aparecen en fórmulas y términos, en particular las variables. Tomemos $t \in \term{S}$, $\varphi \in \form{S}$ tales que existe $V=\{ x_1, \ldots, x_n \} \subseteq \var$ y $\var(\varphi)\subseteq V$ y $\mbox{lib}(\varphi) \subseteq V$. El lema de coincidencia nos permite ignorar el significado de las variables que no están en $V$. Tomamos una interpretación $\mathfrak{Y}=\langle \mathfrak{a}, \, \sigma \rangle$ y $a_i= \sigma(x_i)$ $(1 \leq i \leq n)$. En lugar de tomar $\mathfrak{Y}$, podemos escribir $\mathfrak{a}[a_1/x_1, \ldots, a_n/x_n]$ o simplemente $\mathfrak{a}[\bar{a}/\bar{x}]$ donde $\bar{a}=(a_1, \ldots, a_n)$ y $\bar{x}=(x_1, \ldots, x_n)$
\[ t^{\mathfrak{a}[\bar{a}/\bar{x}]}=t^{\mathfrak{Y}} \qquad \varphi^{\mathfrak{a}[\bar{a}/\bar{x}]}=\varphi^{\mathfrak{Y}} \]
Es particularmente útil para fórmulas cerradas (\textit{i.e.} sin variables libres) en las que podemos tomar $n=0$ ($v=\emptyset$). En ese caso podemos escribir directamente 
\[ \varphi^{\mathfrak{a}} \]
  