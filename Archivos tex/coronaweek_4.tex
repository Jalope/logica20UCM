\subsection{Equivalencias Lógicas}
\begin{definition}
Sean $\varphi, \psi \in \form{S}$. Decimos que $\varphi$ y $\psi$ son \textbf{lógicamente equivalentes}, escrito $\varphi \sim \psi$, si y sólo si para toda interpretación $\mathfrak{Y}$, se verifica $\varphi^{\mathfrak{Y}} = \psi^{\mathfrak{Y}}$ (equivalentemente $\mathfrak{Y} \models \varphi$ si y solo si $\mathfrak{Y} \models \psi$
\end{definition}

\begin{prop}
Sean $\varphi, \psi \in \form{S}$, las siguientes afirmaciones son equivalentes 
\begin{enumerate}
    \item $\varphi \sim \psi$.
    \item $\models \varphi \leftrightarrow \psi$.
    \item $\varphi \models \psi$ y $\psi \models \varphi$.
\end{enumerate}
\end{prop}
\begin{proof}
\textit{Se deja como ejercicio}
\end{proof}

\begin{theorem}[\textbf{Leyes de equivalencia lógica con cuantificadores}] Se tiene que

\begin{enumerate}
	\item Todas las leyes de equivalencia lógica de la lógica proposicional (véase \ref{sec:equivalencias}) siguen siendo validas.
    \item Siendo $u$ variable nueva
    \[\begin{matrix}
    \forall x \, \varphi \sim \forall u \, \varphi[u/x],\\
    \\
    \exists x \, \varphi \sim \exists u \, \varphi[u/x]
    \end{matrix}\]
 
    \item Si $\varphi \sim \psi$, entonces 
\[ \begin{matrix}
\forall x \, \varphi \sim  \forall x \, \psi, \\
\\
\exists x \, \varphi \sim  \exists x \, \psi
\end{matrix} \]    
    
    \item 
\[ \begin{matrix}
\neg \forall x \, \varphi \sim \exists x \, \neg \varphi,\\
\\
\neg \exists x \, \varphi \sim \forall x \, \neg \varphi
\end{matrix} \]    

    \item 
    \[  \begin{matrix}
    \forall x \, \varphi \sim \neg \exists x \, \neg \varphi,\\
    \\
    \exists x \, \varphi \sim \neg \forall x \neg \varphi
\end{matrix}      \]

    \item Si $x \neq y$, entonces 
\[ \begin{matrix}
\forall x \, \forall y \, \varphi \sim \forall y \, \forall x \, \varphi,\\
\\
\exists x \, \exists y \, \varphi \sim \exists y \, \exists x \, \varphi
\end{matrix} \]    
    
    \item 
\[ \begin{matrix} 
\forall x \, \varphi \wedge \psi \sim \forall x \, \varphi \wedge \forall x\, \psi,\\
\\
\exists x \, \varphi \lor \psi \sim \exists x \, \varphi \lor \exists x\, \psi
\end{matrix} \]    
\end{enumerate}
En las reglas que siguen suponemos que $x \notin \mbox{lib}(\psi)$.
\begin{enumerate}[resume]
    \item 
\[ \begin{matrix}
\forall x \, \psi \sim \psi,\\
\\
\exists x \, \psi \sim \psi
\end{matrix} \]    
    
   \item 
\[ \begin{matrix}
(\forall x \, \varphi) \wedge \psi \sim \forall x \, (\varphi \wedge \psi),\\
\\
(\exists x \, \varphi) \lor \psi \sim \exists x \, (\varphi \lor \psi)
\end{matrix} \]   

    \item
\[ \begin{matrix}
\forall x \, (\varphi \lor \psi) \sim \forall x \, \varphi \lor \forall x \, \psi,\\
\\
\exists x \, (\varphi \wedge \psi) \sim \exists x \, \varphi \wedge \exists x \, \psi
\end{matrix} \]    
     
    \item 
\[ \begin{matrix}
(\forall x \, \varphi) \rightarrow \psi \sim \exists x \, (\varphi \rightarrow  \psi),\\
\\
(\exists x \, \varphi) \rightarrow \psi \sim \forall x (\varphi \rightarrow  \psi)
\end{matrix} \]    
\end{enumerate}
\end{theorem}

\begin{proof} 
Se tiene que
\begin{enumerate}
	\item Ya han sido justificadas anteriormente.
	\item Veamos el caso $\forall$, el $\exists$ se hace de forma análoga (\textit{se recomienda escribirlo)}.
\[ \mathfrak{Y} \models \forall u \, \varphi[u/x] \]    
   por definición es, 
   \[ \mbox{para todo } a \in A, \, \mathfrak{Y}[a/u] \models \varphi[u/x] \] 
   Por el lema de sustitución $u^{\mathfrak{Y}[a/u]} = a$, 
\[ \mbox{para todo } a \in A, \, (\mathfrak{Y}[a/u])[a/x] \models \varphi \]   
    puesto que $u$ es nueva, 
\[ (\mathfrak{Y}[a/u])[a/x] \sim_{\varphi} \mathfrak{Y}[a/x] \]    
    por el lema de coincidencia lo anterior es equivalente a, 
\[ \mbox{para todo } a \in A, \, \mathfrak{Y}[a/x] \models \varphi \]    
     que es la definición de $\mathfrak{Y} \models \forall x \, \varphi$.
     
     \item Tenemos $\mathfrak{Y}$  tal que $\mathfrak{Y} \models \exists x \, \varphi$, por definición existe $a \in A$ tal que $\g{Y}[a/x] \models \varphi$ puesto que $\varphi \sim \psi$, lo anterior es equivalente a $a \in A$ con $\g{Y}[a/x] \models \psi$, que es la definición de $\g{Y} \models \exists x \, \psi$. De nuevo el caso $\forall$ es análogo (\textit{Escribirlo}).
     
    \item Tenemos $\g{Y}$ tal que $\g{Y} \models \neg \forall x \, \varphi$ que es lo mismo que $\g{Y} \not \models \forall x \, \varphi$. Por tanto, debe existir $a \in A$ tal que $\g{Y}[a/x] \not \models \varphi$, de forma equivalente $\g{Y} \models \neg \varphi$ que es la definición de $\g{Y} \models \exists x \neg \varphi$. (\textit{Escribir el caso de la negación}).
    
    \item $\varphi \sim \neg \neg \varphi$ por tanto, aplicando (3), $\forall x \, \varphi \sim \forall x \, \neg \neg \varphi$, aplicando (4), $\forall x \, \neg \neg \varphi \sim \neg \exists x \, \neg \varphi$ por tanto 
    \[ \forall x \, \varphi \sim \neg \exists x \, \neg \varphi \] 
    El caso $\exists$ es análogo.
    
    \item Si $x = y$, ver regla (8) más adelante que implica 
    \[ \forall x \forall x \varphi \sim \forall x \varphi \]
    \[ \exists x \exists x \varphi \sim \exists x \varphi \]
    
    \item Sea $\g{Y}$ tal que $\g{Y} \models \forall x \, \varphi \wedge \forall x \, \psi$ es equivalente a que $\g{Y} \models \forall x \, \varphi $ y $\g{Y} \models \forall x \, \psi$. 
    
    Entonces, en el primer caso tenemos que 
\[ \mbox{para todo } a \in A, \, \g{Y}[a/x] \models \varphi \]    
     y en el segundo que 
\[ \mbox{para todo } b \in A, \, \g{Y}[b/x] \models \psi \]     
     obtenemos 
\[ \mbox{para todo } c \in A, \, \g{Y}[c/x] \models \varphi \mbox{ y } \g{Y}[c/x] \models \psi \]     
  entonces 
\[ \g{Y}[c/x] \models \varphi \wedge \psi \]  
finalmente
\[ \g{Y} \models \varphi \wedge \psi \]
    \item Hay que usar que $\g{Y}[a/x] \sim_{\psi} \g{Y}$
 	\item Aplicando (7) y (8).
    \item  
    Hay que demostrar que $\g{Y} \models \forall x\, \varphi \lor \psi$ si y solo si $\g{Y}\models\forall x\,\varphi\lor\forall x\psi$ 
    \begin{itemize}
    	\item[($\Rightarrow$)] $\g{Y}\models\forall x\,(\varphi\lor\psi)$ 
    	\[ \mbox{para todo } a\in A, \, \g{Y}[a/x]\models\varphi\lor\psi \]
    	diferenciamos dos casos 
    	\begin{itemize}
    		\item Existe $a \in A$, $\g{Y}[a/x] \not \models \varphi$ entonces $\g{Y}[a/x]\models\psi$. Por el lema de coincidencia $g{Y} \models \psi$. Por el lema de coincidencia 
    		\[ \mbox{para todo } c \in A, \, \g{Y}[c/x] \models \psi \]
    		que es a definición de \[ \g{Y} \models \forall x \psi \]
    		por tanto $ \g{Y} \models \forall x \varphi \lor \forall x \psi $
    		
    		\item $ \mbox{para todo } a \in A, \, \g{Y}[a/x] \models \varphi $ definición de $\g{Y} \models \forall x \varphi$ 
    	\end{itemize}
    	\item[($\Leftarrow$)] De nuevo dividimos en dos casos 
    	\begin{itemize}
    		\item $\g{Y}\models\forall x \varphi$ sii para todo $a\in A$, $\g{Y}[a/x]\models\varphi$ por definición de $\forall$, para todo $a\in A$, $\g{Y}[a/x]\models\varphi\lor\psi$ por definición de para todo $\g{Y}\models\forall x (\varphi\lor\psi)$
    		\item $\g{Y}\models\forall x \psi$ (\textit{análogo al anterior}.
    	\end{itemize}
    \end{itemize}
     \textit{Escribir los casos para el $\exists$}
\end{enumerate}
\end{proof}

Las leyes de equivalencia permiten \textit{sacar} los cuantificadores al inicio de la fórmula. 

\begin{definition}
Sea $\varphi \in \form{S}$ se dice que está en forma prenexa si es de la forma 
\[ \varphi= Q_1x_1, \ldots, Q_nx_n \; \varphi, \quad Q_i \in \{\forall, \, \exists \} \quad 1 \leq i \leq n \]
$\psi$ (el núcleo) está libre de cuantificadores. 
\end{definition}

\begin{prop}
Sea $\varphi \in \form{S}$. Existe $\psi \in \form{S}$ en forma prenexa tal que $\psi \sim \varphi$ 
\end{prop}
\begin{proof}
Antes de nada observar lo siguiente. Si $x \in \mbox{Lib}(\psi)$
\[ \forall x \varphi \lor \psi \sim \forall u \varphi[u/x] \lor \psi \sim \forall u (\varphi[u/x] \lor \psi) \]
u variable nueva.

La demostración se hace por inducción estructural. Los casos complicados se relación con los conectivos. Se hace para $\{\neg, \, \lor\}$ (los demás casos son análogos, y a la vez, innecesarios pues ya vimos que el conjunto $\{\neg, \, \lor\}$ es funcionalmente completo). 
\begin{itemize}
	\item[($\neg$)] Por hipótesis de inducción $\psi$ en forma prenexa tal que $\varphi \sim \psi$
	\[ \psi= Q_1x_1, \ldots, Q_nx_n \, \psi  \]
	\[ \neg \varphi \sim \neg \psi \sim Q_1'x_1, \ldots, Q_n'x_n \, \psi_1  \]
	donde 
	\[ Q_i'= \left\lbrace \begin{matrix}
\forall & \mbox{si} & Q_i=\exists\\
	\\
	\exists & \mbox{si} & Q_i=\forall
\end{matrix} \right.	 \]
	\item[($\lor$)] Por hipótesis de inducción existen $\varphi'$ y $\psi'$ en forma prenexa tal que 
	\[ \varphi \sim \varphi' \qquad \psi \sim \psi' \]
	\[ \varphi'= Q_1'x_1, \ldots, Q_n'x_n \, \varphi'' \]
	\[ \psi'= Q_1^2y_1, \ldots, Q_m^2y_m \, \psi'' \]
	consideramos variables nuevas $u_1, \ldots, u_n$ y $v_1, \ldots, v_m$, entonces
	\[ \varphi'= Q_1'u_1, \ldots, Q_n'u_n \, \varphi''[\bar{u}/\bar{x}] \]
	\[ \psi'= Q_1^2v_1, \ldots, Q_m^2v_m \, \psi''[\bar{v}/\bar{y}] \]
	puesto que $\bar{u}$ y $\bar{v}$ son nuevas 
	\[ \varphi' \lor \psi'= Q_1'u_1, \ldots, Q_n'u_n \, Q_1^2v_1, \ldots, Q_m^2v_m \, \varphi''[\bar{u}/\bar{x}] \lor \psi''[\bar{v}/\bar{y}]   \]
\end{itemize} 
\end{proof}