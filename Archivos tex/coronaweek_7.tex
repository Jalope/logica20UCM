\subsection{Completitud}
Para poder demostrar la completitud debemos considerar árboles infinitos. ¿Qué es un árbol infinito?
\paragraph{}
\begin{definition}
Un árbol se puede ver como un grafo no dirigido $T=\langle N, \, E \rangle$ donde el conjunto $N$ esta formado por los nodos y el conjunto $E$ por las aristas. Un árbol es un \textbf{grafo conexo  acíclico} (entre cada para de nodos hay una sola secuencia de aristas que los unen. Si a un árbol se le quita una arista dejada de ser un grafo conexo. 
\end{definition}
En un árbol finito se verifica la propiedad $\vert \bar{\tau} \vert = \vert N \vert -1 $ como un si y sólo si. 

\begin{definition}
Dados 2 árboles $\arbol{1}$ y $\arbol{2}$, diremos que $T_1$ está incluido en $T_2$ y escribimos $T_1 \leq T_2$ si 
\begin{itemize}
	\item $N_1 \subseteq N_2$ y la raíz coincide
	\item $E_1 \subseteq E_2$
\end{itemize}
\end{definition}
\paragraph{}
Dada una secuencia de árboles 