\subsection{Semántica}
Vamos a darle un \textit{sentido} a las fórmulas que ya hemos definido. Lo primero que necesitamos es una signatura.

\addtocounter{ej}{1} % sumo 1
\textbf{Ejemplo \Roman{ej}}: dada la signatura $S= \si{ar}$ como
\[ \cts{ar}=\{0 \} \]
\[ \fn{ar}=\{ s\vert_{1},\, +\vert_{2}, \, \ast\vert_{2}\} \]
\[ \pd{ar}=\{ <\vert_{2} \} \]
y llevamos a cabo una escritura usual, en el sentido de que $+(0,s(s(0))) \equiv 0+2$. Esto lo hacemos para todo operador, función o termino con variable. 
\paragraph{}
También vamos a necesitar un soporte y entender como este interacciona con las distintas funciones y predicados.

\addtocounter{ej}{1} % sumo 1
\textbf{Ejemplo \Roman{ej}}:
para las constantes 
\[ \begin{matrix}
&\cts{ar}& \rightarrow & \mathbb{N}\\
&0&\mapsto& 0
\end{matrix} \]
para las funciones
\[ \begin{matrix}
s:&\mathbb{N}& \rightarrow & \mathbb{N}\\
&n&\mapsto& n+1
\end{matrix} \]
\[ \begin{matrix}
+:&\mathbb{N}& \rightarrow & \mathbb{N}\\
&(m,\,n)&\mapsto& m+n
\end{matrix} \] 
\[ \begin{matrix}
\ast : &\mathbb{N}& \rightarrow & \mathbb{N}\\
&(m, \, n)&\mapsto& m \ast n
\end{matrix} \]
finalmente para símbolos de predicados
\[ \begin{matrix}
<:&\mathbb{N}& \rightarrow & \mbox{BOOL}\\
&(m, \, n)&\mapsto& \left \lbrace \begin{matrix}
V & \mbox{si} & m<n\\
F & \mbox{e.c.c.} &
\end{matrix}\right.
\end{matrix} \] 
\paragraph{}
Vamos a dotar a estas funciones y conjuntos de una estructura formal. 
\begin{definition}
Sea $S = \si{S}$ una signatura. Una \textbf{S-álgebra} es una tupla 
\[ \mathfrak{a} := \langle A, \{ c^{\mathfrak{a}} / \, c \in \cts{S} \}, \{f^{\mathfrak{a}} / \, f \in \fn{S}\}, \{p^{\mathfrak{a}} / \, p \in \pd{S} \} \rangle \]
De modo que:
\begin{itemize}
    \item $A \neq \emptyset$, denominado \textbf{conjunto soporte}
    \item Si $c \in \cts{S}$, entonces $c^{\mathfrak{a}} \in A$.
    \item Si $f\vert_k \in \fn{S}$, entonces $f^{\mathfrak{a}}: A^{k} \rightarrow A$.
    \item Si $p\vert_k \in \pd{S}$, entonces $p^{\mathfrak{a}}: A^{k} \rightarrow \mbox{BOOL}$.
\end{itemize}
\end{definition}
\paragraph{}
Como consecuencia, se tiene que 
\begin{enumerate}
	\item $\exists x \, x \doteq x  $ es tautologia
	\item $\forall x \, \neg x \doteq x  $ es contradicción
\end{enumerate}
\paragraph{}
Entonces, lo que hemos dado anteriormente en el ejemplo XXIII es una $S_{ar}$-álgebra. En el siguiente ejemplo, vamos a tomar como soporte un conjunto menos usual que los números naturales. 
\paragraph{}
\addtocounter{ej}{1} % sumo 1
\textbf{Ejemplo \Roman{ej}}: sea $\mathfrak{a} = \langle \{\bigtriangleup, \bigcirc \}, \{0^{\mathfrak{a}}\}, \{+^{\mathfrak{a}}, *^{\mathfrak{a}}, s^{\mathfrak{a}}\}, \{<^{\mathfrak{a}}\}\rangle$ tal que
\begin{itemize}
    \item $0^{\mathfrak{a}} = \bigtriangleup$.
    \item Aplicaciones para cada fórmula y símbolo de proposición 
\[ \begin{matrix}
s^{\mathfrak{a}}:&\{\bigtriangleup, \bigcirc \}& \rightarrow & \{\bigtriangleup, \bigcirc \}\\
&x&\mapsto& \bigtriangleup
\end{matrix} \]

\[ \begin{matrix}
+^{\mathfrak{a}}:&\{\bigtriangleup, \bigcirc \}^2& \rightarrow & \{\bigtriangleup, \bigcirc \}\\
&(x,\,y)&\mapsto&  \bigtriangleup
\end{matrix} \] 

\[ \begin{matrix}
\ast^{\mathfrak{a}} : &\{\bigtriangleup, \bigcirc \}^2& \rightarrow & \{\bigtriangleup, \bigcirc \}\\
&(x, \, y)&\mapsto& \bigcirc 
\end{matrix} \]

\[ \begin{matrix}
<^{\mathfrak{a}}:&\{\bigtriangleup, \bigcirc \}^2& \rightarrow & \mbox{BOOL} \\
&(x, \, y)&\mapsto& V
\end{matrix} \] 
   \end{itemize} 
de forma intuitiva, dadas las dos álgebras para la misma signatura, vemos que tienen \textit{interpretaciones diferentes} a una misma fórmula 
\[ \ast(+(s(0),\,s(0)),\,s(s(s(0)))) \]
\begin{itemize}
	\item Para $\mathbb{N}$ obtenemos 
	\[ \ast(+(1,\,1),\,3)= \ast(2,\,3)=6 \]
	\item Mientras que si tomamos $\{\bigtriangleup, \bigcirc \}$ entonces
		\[ \ast(+(\bigtriangleup,\,\bigtriangleup),\,\bigtriangleup)= \ast(\bigcirc,\,\bigtriangleup)=\bigcirc \] 
\end{itemize} 

Formalicemos el concepto de interpretación mediante la siguiente definición. 
\begin{definition}
Sea $S$ signatura. Una \textbf{S-interpretación} es una tupla $\mathfrak{Y} = \langle \mathfrak{a}, \sigma \rangle$ donde
\begin{itemize}
    \item $\mathfrak{a}$ es una S-álgebra con soporte $A$.
    \item $\sigma : \var \rightarrow A$.
\end{itemize}
\end{definition}
\paragraph{}
Definimos recursivamente la interpretación a través de la siguiente definición
\begin{definition}
Sea $S = \si{S}$ una signatura, $\mathfrak{Y}$ S-interpretación y $t \in \term{S}$.  Sea $t^{\mathfrak{Y}}$
\begin{itemize}
    \item Caso base:
    \begin{itemize}
    	\item $t=c^{\mathfrak{Y}} = c^{\mathfrak{a}}$ si $c \in \cts{S}$
        \item $t=x^{\mathfrak{Y}} = \sigma(x)$ si $x \in \var$
    \end{itemize}
    \item Caso recursivo:
    \begin{itemize}
        \item Si $f\vert_k \in \fn{S}$, $t_1, \dots, t_k \in \term{S}$
        \[ f(t_1, \dots, t_k)^{\mathfrak{Y}} = f^{\mathfrak{a}}(t_{1}^{\mathfrak{Y}}, \dots, t_{k}^{\mathfrak{Y}}) \] 
    \end{itemize}
\end{itemize}
\end{definition}
Del mismo modo, podemos definir recursivamente la \textit{valoración}. Pero antes damos una aclaración en cuestión de notación, abusando del concepto de sustitución definido en la lógica proposicional se tiene que $\sigma[a/x], \, x \in \var, \, a \in A$, $\mathfrak{Y}= \langle \mathfrak{a}, \sigma \rangle$, entonces  $\mathfrak{Y}[a/x]= \langle \mathfrak{a},\sigma[a/x]\rangle$. 
\paragraph{}
\begin{definition}
Sea $S = \si{S}$, $\mathfrak{Y} := \langle \mathfrak{a}, \sigma \rangle$ S-interpretación, con $A$ conjunto soporte. La \textbf{interpretación de una fórmula} se define recursivamente como sigue $\varphi \in \form{S}$, $\varphi^{\mathfrak{Y}}$, como:
\begin{itemize}
    \item Caso base:
        \begin{itemize}
            \item Si $\varphi = \top$, entonces $\varphi^{\mathfrak{Y}} = V$.
            \item Si $\varphi = \bot$, entonces $\varphi^{\mathfrak{Y}} = F$.
            \item Si $\varphi = t \doteq s$, con $t, s \in \term{S}$, entonces 
\[ \begin{matrix}
\varphi^{\mathfrak{Y}} = V & \mbox{si} & t^{\mathfrak{Y}} = s^{\mathfrak{Y}}\\
\varphi^{\mathfrak{Y}} = F & \mbox{e.c.c} 
\end{matrix} \]            
            \item Si $\varphi = p(t_1, \dots, t_k)$, con $t_1, \dots, t_k \in \term{S}$, $p\vert_k \in \pd{S}$, entonces 
            \[ \begin{matrix}
\varphi^{\mathfrak{Y}} = V & \mbox{si} &p^{\mathfrak{a}}(t_{1}^{\mathfrak{Y}}, \dots, t_{k}^{\mathfrak{Y}}) = V\\
\varphi^{\mathfrak{Y}} = F & \mbox{e.c.c} 
\end{matrix} \]  
        \end{itemize}
    \item Caso recursivo:
        \begin{itemize}
            \item Si $\varphi = \neg \varphi_1$, entonces  $\varphi^{\mathfrak{Y}} = v_{\neg}(\varphi_{1}^{\mathfrak{Y}})$
            \item Si $\varphi = \varphi_1 \boox \varphi_2$, entonces $\varphi^{\mathfrak{Y}} = v_{\Box}(\varphi_{1}^{\mathfrak{Y}}, \varphi_{2}^{\mathfrak{Y}})$
            \item Si $\varphi = \forall x \, \varphi_1$, con $x \in \var$, entonces 
            \[ \begin{matrix}
\varphi^{\mathfrak{Y}} = V & \mbox{si} & \mbox{p.t. } a \in A, \, \varphi_1^{\mathfrak{Y}[a/x]} = V\\
\varphi^{\mathfrak{Y}} = F & \mbox{e.c.c} 
\end{matrix} \]  
            \item Si $\varphi = \exists x \, \varphi_1$, con $x \in Var$, entonces
            \[ \begin{matrix}
\varphi^{\mathfrak{Y}} = V & \mbox{si} & \mbox{existe } a \in A, \, varphi_1^{\mathfrak{Y}[a/x]} = V\\
\varphi^{\mathfrak{Y}} = F & \mbox{e.c.c} 
\end{matrix} \]   
        \end{itemize}
\end{itemize}
\end{definition}