%\newcommand{\p}[0]{\mbox{PROP}_{SP}}
%\newcommand{\cb}[0]{\{ \wedge, \, \lor \, \rightarrow, \, \leftrightarrow \}} 
%\newcommand{\boox}[0]{\, \Box \,}

\section*{Lógica de proposiciones}
\begin{definition} Diremos que una \textbf{proposición} es un enunciado que puede ser verdadero o falso. Nunca será una proposición cualquier enunciado que expresa duda o sentimientos. Tampoco lo serán aquellos enunciados que no tengan sentido lógico. 
\end{definition}
\paragraph{}
Un ejemplo de lo que no es proposición sería
\\
p $\equiv$ ``Juan se cae'' 
\\
q $\equiv$ ``Yo me río''
\\
$\varphi \equiv$ Juan se cae y yo me río  
\paragraph{}
\textbf{Conectivas lógicas}. Son los símbolos que utilizamos para formalizar las proposiciones. Estos son 
\[ \lnot \rightsquigarrow \mbox{ negación} \quad \wedge \rightsquigarrow \mbox{ conjunción} \quad 
\lor \rightsquigarrow \mbox{ disyunción} \quad \rightarrow \rightsquigarrow \mbox{ implicación} \quad
 \]  
 \[\leftrightarrow \rightsquigarrow \mbox{ implicación} \quad \bot \rightsquigarrow \mbox{ falso} \quad 
\top \rightsquigarrow \mbox{ cierto} \]

\begin{definition} Se denomina \textbf{formalizar} una proposición, a escribirla mediante conectivas lógicas. 
\end{definition}
\paragraph{}
\newcounter{ej} %creado el contador ejemplo
\addtocounter{ej}{1} % sumo 1
\textbf{Ejemplo \Roman{ej}}: Formalizar las siguientes frases 
\begin{enumerate}
	\item Si llueve se suspende el partido. 
	\item Solo si llueve se suspende el partido.
\end{enumerate}
tomando como proposiciones p $\equiv$ `` Llueve'' y q $\equiv$ ``se suspende el partido''.
\begin{itemize}
	\item[(1)] $ p\rightarrow q$
	\item[(2)] $ q\rightarrow p$
\end{itemize} 
\paragraph{}
\begin{definition} Llamaremos \textbf{formula} a una cadena de símbolos.
\end{definition}

\begin{definition} Denotamos el \textbf{conjunto de} todos los \textbf{símbolos de proposición} como 
	\[ \mbox{SP}=\{p, \, q, \ldots \} \]
	que es un conjunto numerable (no necesariamente finito). 
\end{definition}

\begin{definition} Al conjunto formado por SP y las conectivas lógicas se le denomina \textbf{alfabeto} y lo denotamos como 
	\[ \mbox{A}= \mbox{SP} \cup \{ \lnot, \, \wedge, \, \lor \, \rightarrow, \, \leftrightarrow, \, (, \, ) \} \]
	denotamos por $\mbox{A}^*$ al \textbf{conjunto de cadenas de símbolos} de A
	\[ \mbox{A}^*=\{ \varepsilon, \, a_1, \, a_2, \ldots , a_n : a_n \geq 0,\, a_i \in A,\, 1 \leq j \leq n   \} \]
	donde $\varepsilon$ es la cadena vacía. 
\end{definition}

\addtocounter{ej}{1} % sumo 1
\textbf{Ejemplo \Roman{ej}}: Dado el vocabulario $\mbox{A}=\{ a, \,b \}$ su conjunto de cadena de símbolos será el conjunto 
\[ \mbox{A}^*= \{\varepsilon, \, a, b, ab, ba, aaa, aab, \ldots \} \]
\begin{definition} Dado SP un conjunto de símbolos de proposición, tomamos el alfabeto $\mbox{A}_{SP}$ y definimos $\mbox{PROP}_{SP}$ como el menor subconjunto de $\mbox{A}^*_{SP}$ que verifica 
\begin{enumerate}
	\item $SP \subseteq \mbox{PROP}_{SP}$
	\item Si $\varphi \in \p$, entonces $(\lnot \varphi) \in \p$
	\item Si $\varphi, \psi \in \p$, entonces $(\varphi \, \Box \, \psi) \in \p$, donde \[ \Box \in \cb \] 
\end{enumerate}
\end{definition}
Veamos como se construye esta definición. Sean 
\[ P_0 = \mbox{SP} \]
\[ P_{n+1}= P_n \cup \{(\lnot \varphi), \, (\varphi \, \Box \, \psi) : \, \Box \in \cb, \, \varphi, \psi \in P_n \} \]
\[ P= \bigcup_{i \geq 0} \]
veamos como $P$ cumple las propiedades 1, 2 y 3 de la definición anterior. De forma trivial se verifica que $\p \subseteq P$ y nos faltaría por demostrar la inclusión en el otro sentido. 
\begin{proof}
Sea $\varphi \in P$ entonces $\exists k$ tal que $\varphi \in P_k$ y aplicamos inducción sobre $k$ para ver que $\varphi \in \p$. 
\paragraph{}
Para $k=0$, por la propiedad 1 de la definición se tienen que $\varphi \in \p$. Para $k \geq 0$ 
\begin{itemize}
	\item[(i)] $\varphi \in P_{k-1}$
	\item[(ii)] $\psi \in P_{k-1}$ tal que $\varphi = (\lnot \psi)$
	\item[(iii)] $\psi_1, \psi_2 \in P_{k-1}$ entonces $\varphi = (\psi_1 \boox \psi_2)$
\end{itemize}
\end{proof}